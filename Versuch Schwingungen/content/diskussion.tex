\section{Diskussion}
\label{sec:Diskussion}
    Die meisten Fehler, die in diesem Versuch auftreten, liegen im Toleranzbereich.
    Neben den Fehlern, die die einzelnen Bauteile aufweisen, ist hier vor allem die Ungenauigkeit, die beim Ablesen des Oszillographenentstehen, ausschlaggebend.
    Durch diese lassen sich nicht nur die Fehler auf die Amplituden und Phasen erklären, die in weiteren Rechnungen auch die Parameter $U_0$und $2\symup{\pi}\cdot \mu$ sowie die daraus berechnete Abklingdauer $T_{ex}$ in Aufgabenteil a)  beeinflussen, sondern sie führt auch zueinem Fehler auf den abgelesen Widerstand $R_{ap}$ in b), da dieser nicht so eingestellt werden kann, dass exakt der aperiodische Grenzfallvorliegt.
    In Teil c) liegt die gemesse Kondensatorspannung mit ihrer Resonanzfrequenz sehr weit von der errechneten weg. Grund dafür kann erneut dasungenaue Messen bzw. die Fehler der Bauteile sein, jedoch ist es unwahrscheinlich, dass daraus eine derart drastische Abweichung zustandekommt. Vermutlich liegt daher der Fehler eher in den Rechnungen.
    In Teil d) nähert die Kurve für $\Delta \phi$ die Messwerte relativ gut an. Die Abweichung an dieser Stelle ist höchst wahrscheinlich durchFehler, bzw. Ungenauigkeiten beim Messen aufgrund der begrenzten Genauigkeit der Skala der Messgeräte, zu erklären.